
\documentclass{article}
\input{myinclude}
\usepackage[numbers]{natbib}

\begin{document}

\title{Mercel Kernel}
\maketitle

\section*{Description}

A Mercer Kernel is a \href{\kmlroot/kernel.html}{Kernel} that is positive definite.

Positive definiteness in the context of kernel machines implies that a kernel matrix created using 
a particular kernel is positive definite. A matrix is positive definite if its associated eigenvalues
$\lambda_1,\lambda_2,\ldots,\lambda_N$ are positive.


\section*{Refinement of}

\href{http://www.sgi.com/tech/stl/AdaptableBinaryFunction.html}{Adaptable Binary Function}, 
\href{http://www.sgi.com/tech/stl/DefaultConstructible.html}{Default Constructible}, 
\href{http://www.boost.org/doc/html/CopyConstructible.html}{Copy Constructible},
\href{http://www.sgi.com/tech/stl/Assignable.html}{Assignable}

\section*{Associated types}

\section*{Notation}
\section*{Definitions}
\section*{Valid Expressions}
\section*{Expression Semantics}

\section*{Complexity Guarantees}

\section*{Invariants}

\section*{Models}

\begin{itemize}
\item \href{\kmlroot/linear.html}{linear}
\item \href{\kmlroot/gaussian.html}{gaussian}
\item \href{\kmlroot/hermitian.html}{hermitian}
\item \href{\kmlroot/polynomial.html}{polynomial}
\end{itemize}

\section*{Notes}


\section*{See also}


\bibliographystyle{unsrtnat}
\bibliography{/home/rutger/documents/bibliography/references}
\end{document}



