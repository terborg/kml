
\documentclass{article}
\input{myinclude}
\usepackage[numbers]{natbib}

\begin{document}

\title{Regression}
\maketitle

\section*{Description}

Generally, a regression problem is to find a relation between variables. 
In machine learning, it is considered a supervised learning problem, where we 
learn from instances from a data set 
$\mathcal{D}=\{ (x_1,y_1), (x_2,y_2), \ldots, (x_n, y_n ) \}$ 
containing input-output pairs $(x_i,y_i)\in \mathcal{X}\times \mathbb{R}$. 

\begin{figure}
\fignomath{regression_problem}
\end{figure}


\section*{Refinement of}

(is this part of this concept?)
\href{http://www.boost.org/libs/property_map/LvaluePropertyMap.html}{LvaluePropertyMap}

\section*{Associated types}
\section*{Notation}
\section*{Definitions}
\section*{Valid Expressions}
\section*{Expression Semantics}
\section*{Complexity Guarantees}
\section*{Invariants}
\section*{Models}
\section*{Notes}
\section*{See also}

\href{\kmlroot/classification.html}{Classification},
\href{\kmlroot/ranking.html}{Ranking}

\bibliographystyle{unsrtnat}
\bibliography{/home/rutger/documents/bibliography/references}
\end{document}



