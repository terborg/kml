
\documentclass{article}
\input{myinclude}
\usepackage[numbers]{natbib}

\newcommand{\half}{\tfrac{1}{2}}

\begin{document}

\title{gaussian<T>}
\maketitle

\section*{Description}

\texttt{Gaussian<T>} is a Kernel. Specifically, it is a \href{\kmlroot/reference/mercer_kernel.html}{Mercer Kernel}. If k is an object of class \texttt{gaussian<T>}, and \texttt{u} and \texttt{v} are objects of class T, then k(u,v) returns
%
$$k(u,v) = \textrm{exp}( - \half \sigma^{-2} \Vert u-v \Vert^2 ) $$
%
where $\sigma$ is the width of the kernel. 
Figure~\ref{figure:gaussian_kernel} shows a Gaussian kernel.

\begin{figure}
\includegraphics{gaussian_kernel}
\includegraphics{gaussian_kernel_3d}
\caption{A two dimensional (left) and three-dimensional (right) plot of a single Gaussian kernel located at the origin.}
\label{figure:gaussian_kernel}
\end{figure}

The Gaussian kernel is a good choice for a great deal of applications, although sometimes it is remarked as being
overused \cite{scholkopf02learning}.


\section*{Example}


\highlightcpp{}
\begin{verbatim}
std::vector< double > u(10);
std::vector< double > v(10);
gaussian< std::vector< double > > kernel(1.0);
std::cout << kernel( u, v ) << std::endl;
\end{verbatim}


\section*{Definition}

Defined in the KML header \href{\kmlsvnroot/kml/gaussian.hpp}{kml/gaussian.hpp}.


\section*{Template Parameters}

\begin{tabular}{lll}
\textbf{Parameter} & \textbf{Description} & \textbf{Default} \\ 
\hline
T & The gaussian argument type \\ 
\end{tabular}


\section*{Model of}

\href{\kmlroot/reference/mercer_kernel.html}{Mercer Kernel}

\section*{Type requirements}
T must be a vector type or a numeric type; distance_squared<T> should evaluate.

\section*{Members}

\begin{tabular}{lll}
\textbf{Member} & \textbf{Where defined} & \textbf{Description} \\ 
\hline
\texttt{gaussian()} & \href{http://www.sgi.com/tech/stl/DefaultConstructible.html}{Default Constructible} & The default constructor \\
\texttt{result_type} & Input value & The type of the result: \texttt{input_value<T>} \\
\end{tabular}

\section*{Notes}

\section*{See also}

\href{\kmlroot/reference/mercer_kernel.html}{Mercer Kernel},
\href{\kmlroot/reference/linear.html}{linear},
\href{\kmlroot/reference/hermitian.html}{hermitian},
\href{\kmlroot/reference/polynomial.html}{polynomial},
\href{\kmlroot/reference/sigmoid.html}{sigmoid}

\bibliographystyle{unsrtnat}
\bibliography{/home/rutger/documents/bibliography/references}
\end{document}

